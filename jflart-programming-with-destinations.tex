\documentclass[english]{jflart}
\usepackage[utf8]{inputenc}
\usepackage[T1]{fontenc}
\usepackage{minted}
\usepackage{etoolbox,xpatch}
\usepackage{ tipa }
\usepackage[normalem]{ulem}
\usepackage[backend=biber, style=alphabetic]{biblatex}
\addbibresource{bibliography.bib}

% Numéro et année des JFLAs visées par l'article, obligatoire.
\jfla{35}{2024}

\title{Programming with destinations in Haskell}
% Un titre plus court, optionnel.
%\titlerunning{Du bon usage de~\texttt{jflart.cls}}

% Auteurs, liste non abrégée.
\author[1]{Thomas Bagrel}
% \author[2]{Cunégonde Martin}
% \author[2]{Odoacre Contempierre}
% Une liste d'auteurs abrégée à utiliser à l'intérieur de l'article.
\authorrunning{Bagrel}

% Affiliations des auteurs
\affil[1]{INRIA/LORIA, Vand\oe{}uvre-lès-Nancy, 54500, France}
\affil[2]{TWEAG, Paris, 75012, France}

% Une commande définie par l'utilisateur
\newcommand{\cmd}[1]{\texttt{\textbackslash {#1}}}
\newcommand{\mpar}{\text{\,\textramshorns\,}}
\newcommand{\dest}{-\prec}
\usepackage{newunicodechar}
\newunicodechar{⊸}{\ensuremath{\multimap}}
\newunicodechar{→}{\ensuremath{\to}}
\newunicodechar{⇒}{\ensuremath{\Rightarrow}}
\newunicodechar{∀}{\ensuremath{\forall}}
\makeatletter
\AtBeginEnvironment{minted}{\dontdofcolorbox}
\def\dontdofcolorbox{\renewcommand\fcolorbox[4][]{##4}}
\xpatchcmd{\inputminted}{\minted@fvset}{\minted@fvset\dontdofcolorbox}{}{}
\xpatchcmd{\mintinline}{\minted@fvset}{\minted@fvset\dontdofcolorbox}{}{} % see https://tex.stackexchange.com/a/401250/
\makeatother

\begin{document}

\maketitle

\begin{abstract}
Destination-passing style programming introduces destinations, which represents the address of a write-once memory cell. Those destinations can be passed around as function parameters, and thus enable the caller of a function to keep control over memory allocation: the body of the called function will just be responsible of filling that memory cell. This is especially useful in functional programming languages such as Haskell, in which the body of a function is typically responsible for allocation of the result value.

Programming with destination in Haskell is an interesting way to improve performance of critical parts of some programs, without sacrificing memory warranties. Indeed, thanks to a linearly-typed API we designed, a write-once memory cell cannot be left uninitialized before being read, and is still disposed of by the garbage collector when it is not in use anymore, eliminating the risk of uninitialized read, memory leak, or double-free errors that can arise when memory is handled manually.

With the implementation of destinations for Haskell through compact regions we provide in this article, we reach a 15-40\% improvement over memory allocation in a simple parser example, and 0-50\% improvement in run time. We also provide a few examples of programs that can be implemented in a tail-recursive fashion thanks to destinations, which is crucial for performance in strict contexts.

Safety proofs for the API are not provided in this article though, and will be the subject of a future article.
\end{abstract}

\tableofcontents{}

\section{Introduction}

TODO: conceptuellement simple, mais beaucoup d'obstacles techniques pour rendre ça possible.

 Destinations bring a taste of imperative programming in a pure functional environnement when performance really matters, without breaking memory safety.

Using destinations as a way of allocating and building functional data structures can lead to better time and/or space performance for critical parts of a program, but destinations also increase expressiveness of a functional language.

\subsection{Problem space and motivation}

\subsection{Framework}

- GHC where purity is enforced

- Also because it has good support for linear type discipline

TODO: rappel théorique

- Linear types
- Destinations
- Compact regions
- strict lazy haskell

(préliminaire car ça existe déjà)

Article à destination des experts haskell ? CamL?

\section{Motivating examples for destination-style programming}

\subsection{(Perf) Efficient Difference-list implementation}

\subsection{(Expressiveness) BFS Tree mapping/traversal}

Traversing a binary tree in a breadth-first fashion with monadic effects is notoriously hard to implement in a pure functional fashion. Indeed, when a node in the resulting tree is produced, we cannot build its children immediately because the effects of the children node creation would break the BFS order. Instead, we must store the path to that node in a queue, and come back much later to add its children, so that BFS effect order is preserved.

With destinations in our toolbelt, the problem is much simpler, as we can just let that node be "incomplete" for the time being, store the destinations for its children in a queue, and fill them later when they reach the head of the queue. There is no need to store the path up to that node in a fancy fashion, and no need to build many temporary functional structures in memory either. In a way, destinations provide a single-use \emph{action-at-a-distance} concept.

TODO: parler d'Ur dans les types linéaires
TODO: parler d'alloc brièvement; buildInRegionAndExtract

\begin{minted}[linenos]{haskell}
data Tree a = Nil | Node a (Tree a) (Tree a)

mapStateBFS :: ∀ a b s. (s → a → (s, b)) → s → Tree a → (s, Tree b)
mapStateBFS f s0 tree =
  buildInRegionAndExtract $
    \token → alloc token <&>
      \dtree → go s0 (singleton (Ur tree, dtree))
  where
    go :: s → Queue (Ur (Tree a), Dest (Tree b)) ⊸ Ur s
    go s q = case dequeue q of
      Nothing → Ur s
      Just ((utree, dtree), q') → case utree of
        Ur Nil → dtree & fill @'Nil `lseq` go s q'
        Ur (Node x tl tr) → case dtree & fill @'Node of
          (dr, dtl, dtr) →
            let q'' = q' `enqueue` (Ur tl, dtl) `enqueue` (Ur tr, dtr)
                (s', r) = f s x
              in dr & fillLeaf r `lseq` go s' q''
\end{minted}

\subsection{(Perf) Deserializing from a structured format (S-Expr)}

Derserialization is a very common task in application development, and can become a crucial part of the program performance-wise, when the program in question is meant to be fed with large chunks of serialized data.

Most of the time, the result of deserialization will be a data tree, with hundred of nodes and leaves, whose consumption by the processing program will be strict because:
\begin{itemize}
  \item the programmer wants to catch any potential error in the input document early enough in the process;
  \item most of the input will be read to produce the output anyway.
\end{itemize}

Also, it's uncommon for a part of the input (in its parsed but unprocessed form) to greatly outlive the rest of it. As a result, treating the deserialized data as a large object with a unique lifetime for all its parts is a decent  approximation when that allows easier memory management, as we will show in a next section.

The current implementation of destination-style programming for Haskell is based on Compact Regions, which bring to the table the two aforementioned properties: strictness, and a shared lifetime for a tree of objects (which may keep some nodes in memory for slightly longer than necessary), in exchange of impressive garbage collection savings!

Let's focus on a deserializer for S-expressions. S-expresssions are parenthesized lists whose elements are just seperated by spaces. These elements can be of several types: Int, String, Symbol (a textual token, with no quotes around it, unlike String), or (nested) list.

Parsing a S-expression can be done concisely with three mutually recursive functions:
\begin{itemize}
  \item \mintinline{haskell}`parseSExpr` scans the next character, and either dispatches to \mintinline{haskell}`parseSList` if it encounters an opening parenthesis, or to \mintinline{haskell}`parseSString` if it encounters an opening quote, or eventually parses the string into a number or symbol;
  \item \mintinline{haskell}`parseSList` calls \mintinline{haskell}`parseSExpr` to parse the next token, and then calls itself again until reaching a closing parenthesis, accumulating the parsed elements along the way;
  \item \mintinline{haskell}`parseSString` scans the input character by character and accumulates them until reaching a closing quote (taking escape sequences into consideration).
\end{itemize}

\begin{minted}[linenos]{haskell}
parseSExpr :: ByteString → Int → Either ParseError SExpr
parseSExpr bs i = case bs !? i of
  Nothing → Left $ UnexpectedEOFSExpr i
  Just c → case c of
    ')' → Left $ UnexpectedClosingParen i
    '(' → parseSList bs (i + 1) []
    '"' → parseSString bs (i + 1) False []
    _ →
      let tok = extractNextToken bs i -- take chars until delimiter/space
       in if null tok
            then -- c is a (leading) space, skip it and recurse
              parseSExpr bs (i + 1)
            else case parseInt tok of
              Just int → Right $ SInteger (i + length tok - 1) int
              Nothing → Right $ SSymbol (i + length tok - 1) (toString tok)

parseSList :: ByteString → Int → [SExpr] → Either ParseError SExpr
parseSList bs i acc = case bs !? i of
  Nothing → Left $ UnexpectedEOFSList i
  Just c → \cases
    | c == ')' → Right $ SList i (reverse acc)
    | isSpace c → parseSList bs (i + 1) acc
    | otherwise → case parseSExpr bs i of
        Left err → Left err
        Right child → parseSList bs (endPos child + 1) (child : acc)

parseSString :: ByteString → Int → Bool → [Char] → Either ParseError SExpr
parseSString bs i escape acc = case bs !? i of
  Nothing → Left $ UnexpectedEOFSString i
  Just c → case c of
    '"' | not escape → Right $ SString i (reverse acc)
    '\\' | not escape → parseSString bs (i + 1) True acc
    'n' | escape → parseSString bs (i + 1) False ('\n' : acc)
    _ → parseSString bs (i + 1) False (c : acc)
\end{minted}

TODO: préciser slicing?

Now, this implementation is already quite efficient for a strict setting. The functions are as tail-recursive as they can be, and they use indexing and slicing into a bytestring ($\mathcal{O}(1)$ operation) instead of potentially allocating a huge number of strings on the heap.

That being said, it is possible to obtain very significative performance gains by using destinations with only very little stylistic changes in the code.

Accumulators of tail-recursive functions just have to be changed into destinations. Instead of writing elements into a list that will be reversed at the end as we did before, the program in the destination style will directly write the elements into their final location.

It is good to note that destination allow to reverse the natural order in which a structure is built. For a list, the natural operation in functional programming languages is \emph{prepend}/\emph{cons}, which adds an element at the front of an existing list (bottom-up approach). Thanks to destinations, it's possible to build a list starting from an element which will stay at the head of the it, and new elements will be added towards the tail of the list (\emph{append}/\emph{fillCons}). It's entirely possible to mix both approaches too, as it will be detailed in the API section.

\newcommand{\mnew}[1]{\colorbox{green}{#1}}
\newcommand{\mold}[1]{\colorbox{red}{\sout{#1}}}

\begin{minted}[escapeinside=°°,linenos]{haskell}
parseSExprDS :: ByteString → Int → Dest SExpr ⊸ Either ParseError Int
parseSExprDS bs i d = case bs !? i of
  Nothing → °\mnew{d & fillLeaf defaultSExpr}° `lseq` Left $ UnexpectedEOFSExpr i
  Just c → case c of
    ')' → °\mnew{d & fillLeaf defaultSExpr}° `lseq` Left $ UnexpectedClosingParen i
    '(' → parseSListDS bs (i + 1) °\mnew{(d & fill @'SList)}\mold{[]}°
    '"' → parseSStringDS bs (i + 1) False °\mnew{(d & fill @'SString)}\mold{[]}°
    _ →
      let tok = extractNextToken bs i -- take chars until delimiter/space
        in if null tok
            then -- c is a (leading) space, skip it and recurse
              parseSExprDS bs (i + 1) d
            else case parseInt tok of
              Just int →
                °\mnew{d & fill @'SInteger & fillLeaf int}°
                  `lseq` Right (i + length tok - 1)
              _ →
                °\mnew{d & fill @'SSymbol & fillLeaf (toString tok)}°
                  `lseq` Right (i + length tok - 1)

parseSListDS :: ByteString → Int → Dest [SExpr] ⊸ Either ParseError Int
parseSListDS bs i d = case bs !? i of
  Nothing → °\mnew{d & fill @'[]}° `lseq` Left $ UnexpectedEOFSList i
  Just c →
    \cases
      | c == ')' → °\mnew{d & fill @'[]}° `lseq` Right i°\mold{(reverse acc)}°
      | isSpace c → parseSListDS bs (i + 1) d
      | otherwise →
          let !(dh, dt) = °\mnew{d & fill @'(:)}°
          in case parseSExprDS bs i °\mnew{dh}° of
                Left err → dt & fill @'[] `lseq` Left err
                Right endPos → parseSListDS bs (endPos + 1) °\mnew{dt}\mold{(child : acc)}°

parseSStringDS :: ByteString → Int → Bool → Dest [Char] ⊸ Either ParseError Int
parseSStringDS bs i escape d = case bs !? i of
  Nothing → °\mnew{d & fill @'[]}° `lseq` Left $ UnexpectedEOFSString i
  Just c → case c of
    '"' | not escape → °\mnew{d & fill @'[]}° `lseq` Right i°\mold{(reverse acc)}°
    '\\' | not escape → parseSStringDS bs (i + 1) True d
    'n' | escape →
        let !(dh, dt) = °\mnew{d & fill @'(:)}°
         in °\mnew{dh & fillLeaf '\textbackslash{}n'}° `lseq` parseSStringDS bs (i + 1) False °\mnew{dt}\mold{('\textbackslash{}n' : acc)}°
    _ →
        let !(dh, dt) = °\mnew{d & fill @'(:)}°
         in °\mnew{dh & fillLeaf c}° `lseq` parseSStringDS bs (i + 1) False °\mnew{dt}\mold{(c : acc)}°
\end{minted}


\begin{itemize}
  \item Even for error cases, we are forced to consume the destination that we receive as an argument, hence we write some sensible default data to it (see lines 3, 5, 23 and 36)
  \item Because of the top-down approach, it's necessary to choose which variant of SExpr is being built very early (by writing the corresponding constructor into the \mintinline{haskell}`SExpr` destination), instead of building an accumulator and wrapping the variant constructor over it as it was done in the naive implementation (see lines 6 and 7);
  \item The SExpr value resulting from the call of \mintinline{haskell}`parseSExpr` inside \mintinline{haskell}`parseSList` is no longer collected by the caller; but instead written directly into its final location by the callee (see lines 30 to 32).
  \item Adding an element \mintinline{haskell}`a` to the accumulator \mintinline{haskell}`[a]` is replaced with adding a new cons cell with \mintinline{haskell}`fill @'(:)`, writing the element to the head destination (\mintinline{haskell}`Dest a`), and then continuing the processing with the tail destination (\mintinline{haskell}`Dest [a]`) (see lines 29, 32, 42 and 45).
  \item Instead of reversing and returning the accumulator at the end of the processing, it is enough to complete the list by writing a nil element to the tail destination (with \mintinline{haskell}`fill @'[]`) (see lines 26 and 38).
\end{itemize}

- More efficient GC-wise (see -T benchmark)

- More efficient runtime-wise

\section{Technical development}

\subsection{API Design}

A destination represents a hole in a data structure, and allows for that hole to be filled even if the 

The main design principle behind destination-style structure building is that no structure can be read before all its destinations have been written to. That way, incomplete data structures can be freely passed around and stored, but need to be completed before any pattern-matching can be made on them.

Hence we introduce a new data type \mintinline{haskell}`Incomplete r a b` where \mintinline{haskell}`a` stands for the type of the structure being built, and \mintinline{haskell}`b` is the type of what needs to be linearly consumed before the structure can be read.

Types aren't linear by themselves in Linear Haskell. Instead, functions can be made linear or not, and linearity of resources are ensured through scope-functions: functions taking a callback that linearly consumes the resource (very much like continuation-passing style).

TODO: Il faut que le seul moyen d'avoir resource en position positive soit en CPS avec callback linéaire (ou dans des fonctions qui ont une resource en position négative aussi)

Let's make things clearer with an example:
\begin{minted}{haskell}
data Resource

withResource :: (Resource ⊸ a) ⊸ a
\end{minted}

If \mintinline{haskell}`withResource` is the only producer of \mintinline{haskell}`Resource`, then the only way to ever access a resource will be to supply a linear callback to \mintinline{haskell}`withResource`. Still, this is not enough; because \mintinline{haskell}`\x → x` is indeed a linear callback, one could use \mintinline{haskell}`withResource (\x → x)` to leak a \mintinline{haskell}`Resource`, and then use it in a non-linear fashion.

We must force the callback to actually consume the resource, and not leak it to the outside. To forbid the resource from appearing anywhere in the return type of the callback, we will ask the return type to be wrapped in \mintinline{haskell}`Ur`. Putting something in \mintinline{haskell}`Ur` is a non-linear operation, except for \mintinline{haskell}`Movable` types, which are basic ones (\mintinline{haskell}`Int`, \mintinline{haskell}`String`, \mintinline{haskell}`Char`...) and structures made of them. As linear resource is simply a data structure which doesn't implement \mintinline{haskell}`Movable`, and which cannot be wrapped linearly in \mintinline{haskell}`Ur`.

With the following declaration, our linear resource cannot leak to the outside world, and must be consumed linearly by the callback (using other functions supplied by the API, written in direct style this time):

\begin{minted}{haskell}
class Movable a where
  move :: a ⊸ Ur a

data Resource

withResource :: (Resource ⊸ Ur a) → Ur a
updateResource :: Int ⊸ Resource ⊸ Resource
closeResource :: Resources ⊸ ()

-- OK
withResource (\r → r & updateResource 42 & closeResource & move) :: Ur ()

-- fails with linearity error
withResource (\r → r & move) :: Ur Resource
\end{minted}

This is mostly the design principle that have been used for destinations in Haskell. In order to access the \mintinline{haskell}`Incomplete`'s \mintinline{haskell}`a` value, the \mintinline{haskell}`b` side must be transformed/consumed into something with type \mintinline{haskell}`Ur c`. Because the \mintinline{haskell}`b` side hosts the destinations initially, they have to be consumed by a linear callback mapping on \mintinline{haskell}`b` side before \mintinline{haskell}`fromReg` can be used to access the \mintinline{haskell}`a`. As we explained above, they cannot leak as there is no linear way to produce a \mintinline{haskell}`Ur Dest` from a \mintinline{haskell}`Dest`.

\begin{minted}{haskell}
newtype Incomplete r a b = Incomplete (a, b)

instance Control.Functor (Incomplete r a) where
  fmap :: (b ⊸ c) ⊸ Incomplete r a b ⊸ Incomplete r b c
  fmap f (Incomplete (s, d)) = Incomplete (s, f d)

fromRegExtract :: ∀ r a b. (RegionContext r) ⇒ Incomplete r a (Ur b) ⊸ Ur (a, b)
\end{minted}

\mintinline{haskell}`Region`, \mintinline{haskell}`RegionContext r` and \mintinline{haskell}`RegionToken r` are mostly implementation noise for the API. There presence will be justified in the next subsection.

Allocation a new receiver of type \mintinline{haskell}`a` is done through \mintinline{haskell}`alloc`:

\begin{minted}{haskell}
alloc :: ∀ r a. RegionToken r ⊸ Incomplete r a (Dest r a)
\end{minted}

This function signature can be read that way : it consumes a region token, and returns an object holder, which upon consumption of a \mintinline{haskell}`Dest r a`, will unlock the object of type \mintinline{haskell}`a`. At this point, the return value of `alloc` is pretty much the identity: give it an \mintinline{haskell}`a` (to fill the \mintinline{haskell}`Dest r a`), and it will return an \mintinline{haskell}`a`.

To fill that hole represented by \mintinline{haskell}`Dest r a`, several functions are available:

\begin{itemize}
  \item \mintinline{haskell}`fillLeaf :: ∀ r a. (RegionContext r) ⇒ a → Dest r a ⊸ ()` \\will use a non-linear value of type \mintinline{haskell}`a` to fill the hole;
  \item \mintinline{haskell}`fillComp :: ∀ r a b. (RegionContext r) ⇒ Incomplete r a b ⊸ Dest r a ⊸ b` \\will use an incomplete object of type \mintinline{haskell}`a` to fill the hole; and the resulting holes for the bigger structure will be the ones of that object. In other terms, \mintinline{haskell}`fillComp` composes two structures with holes together;
  \item \mintinline{haskell}`fill :: ∀ liftedCtor r a. Dest r a ⊸ DestsOf liftedCtor r a` \\takes a constructor as a type parameter (\mintinline{haskell}`liftedCtor`) and will write a shallow instance of that constructor into the \mintinline{haskell}`Dest r a`, returning the destinations corresponding to that constructor's fields.\\ For example, \mintinline{haskell}`fill @Just @r @(Maybe Int) :: Dest r (Maybe Int) ⊸ Dest r Int` writes a shallow \mintinline{haskell}`Just` constructor in the \mintinline{haskell}`Maybe Int` hole, and returns the \mintinline{haskell}`Dest Int` corresponding to the hole of type \mintinline{haskell}`Int` that still needs to be filled.
\end{itemize}

\mintinline{haskell}`fill` is probably the most interesting of the three, and will be the most used one too, because it enables the user to build data structures in a top-down approach, which complements very well the natural bottom-up way of constructing data structures in functional programming languages. Thanks to destinations, the user can now choose between those two approaches and pick the most efficient or more natural way for the problem at hand.

\mintinline{haskell}`fillComp` offers a way to mix both approaches: one can build small chunks in a top-down approach, and combine them together in a bottom-up fashion even if they aren't all complete.

\mintinline{haskell}`fillLeaf` is just a restriction of \mintinline{haskell}`fillComp`. It is used very frequently for types whose constructors aren't \mintinline{haskell}`Fill`able because they wrap over unpacked or primitive fields.

\subsection{Note about Compact Regions}

At the moment, destination-style programming is only possible in compact regions. Compact regions are special chunks on the heap that will only be very lightly inspected by the garbage collector, and thanks to that, we can do chirurgical memory updates in those without being afraid that it will interfere with garbage collection (especially move operations).

Because we have immobile chunks of memory, destinations can be implemented as a wrapper over a raw pointer which points to the memory location where data have to be written:

\begin{minted}{haskell}
data Dest r a = Dest Addr#
\end{minted}

The phantom type parameter \mintinline{haskell}`r` that is present everywhere in the API ensures that objects can only interact with other ones from the same region (as outgoing pointers across different regions are not allowed by design of Compact regions).

A \mintinline{haskell}`Region` object is implemented as a newtype over \mintinline{haskell}`Compact FirstInhabitant`, which is composed of a pointer to the region header, a lock (because compact region writes are not thread-safe at the primitive level), and a pointer to a \mintinline{haskell}`FirstInhabitant` inside the region.

\mintinline{haskell}`RegionToken r` carries a non-linear \mintinline{haskell}`Region` and is used when a \mintinline{haskell}`Proxy r` would be necessary anyway to infer the \mintinline{haskell}`r` of \mintinline{haskell}`Dest`/\mintinline{haskell}`Incomplete` in a function signature. \mintinline{haskell}`RegionContext r` makes the region argument implicit when the region identifier \mintinline{haskell}`r` is already part of the types of the function arguments.

\subsection{User-land haskell implementation details}
 
One issue I had during implementation was choosing the right representation for \mintinline{haskell}`a` in \mintinline{haskell}`Incomplete r a b`.

Ideally, we want \mintinline{haskell}`Incomplete r a b` to contains a \mintinline{haskell}`a` and a \mintinline{haskell}`b`, and let the \mintinline{haskell}`a` free when the \mintinline{haskell}`b` is fully consumed (or linearly transformed into \mintinline{haskell}`Ur c`). So the most straightforward representation of Incomplete would be a pair \mintinline{haskell}`data Incomplete r a b = Incomplete a b`.

It is also natural for \mintinline{haskell}`alloc` to return an \mintinline{haskell}`Incomplete r a (Dest a)`: as soon as the \mintinline{haskell}`Dest a` is linearly consumed, a \mintinline{haskell}`a` will be available.

However, building something of type \mintinline{haskell}`Incomplete r a (Dest a)` is not easy. The \mintinline{haskell}`Dest a` in question should carry the address of a memory cell in which the address of a constructor of type \mintinline{haskell}`a` will be written (by \mintinline{haskell}`fill` or a similar function). That memory cell is named \emph{root receiver}, because it will receive the address of the root of the object being built. Given the data definition above, the root receiver should hence be the first field of the \mintinline{haskell}`Incomplete` object. But the Incomplete object is not part of the compact region itself! It lives in the normal GC heap (which is crucial for early deallocation and memory optimizations); so it may move. As a result, we cannot just use the address of its left field as a destination and the left field as a root receiver.

The next natural move is to allocate a wrapper \mintinline{haskell}`W a` for \mintinline{haskell}`a` in the region, that will act as a root receiver, and have \mintinline{haskell}`data Incomplete r a b = Incomplete (W a) b`. Because the \mintinline{haskell}`a` will have to be wrapped in \mintinline{haskell}`Ur` when it is complete (see \mintinline{haskell}`fromReg :: ∀ r a. (RegionContext r) ⇒ Incomplete r a () ⊸ Ur a`), we might as well use \mintinline{haskell}`Ur` as a wrapper and declare \mintinline{haskell}`data Incomplete r a b = Incomplete (Ur a) b`. It solves the previous issue of not having a receiver to write into.

Unfortunately, that approach is quite unsatisfying because any \mintinline{haskell}`Incomplete` object will now allocate a few words into the region that won't be collected for a long time. In particular, putting a non-linear value inside the region with \mintinline{haskell}`intoReg :: ∀ r a. RegionToken r ⊸ a → Incomplete r a ()` will allocate an unnecessary \mintinline{haskell}`Ur` wrapper, as the object is already fully built, and thus doesn't need a root receiver. With the \mintinline{haskell}`Ur` solution, it's no longer efficient to write \mintinline{haskell}`fillLeaf` as a composition \mintinline{haskell}`fillComp . intoReg (getToken @r)`, and \mintinline{haskell}`fillLeaf` needs a dedicated optimized implementation.

Another approach we explored is to represent an \mintinline{haskell}`Incomplete` as a piece of data structure that would be concretized when it is given a root receiver. Delaying the construction of the structure that way is not efficient though, because instead of writing a structure to memory, we create huge closures representing the action of creating the structure.

Finally, I decided to embed the \mintinline{haskell}`a` field in \mintinline{haskell}`Incomplete` not in a regular data constructor wrapper, but instead in a STG indirection closure (\mintinline{haskell}`stg_IND` symbol). The indirection closure is allocated into the compact region in the \mintinline{haskell}`alloc` case (for the same reasons as the \mintinline{haskell}`Ur` above was), but no indirection closure is needed in \mintinline{haskell}`intoReg`, we can just store the fully built \mintinline{haskell}`a` directly in \mintinline{haskell}`Incomplete`. That way, we keep uniformity in the typing and runtime behaviour of \mintinline{haskell}`Incomplete` in both cases (as indirections are transparent for the RTS), and ensure that the overhead of `fillLeaf`/\mintinline{haskell}`intoReg` is very small, at the cost of a few more alloc-ed words for each call of \mintinline{haskell}`alloc`. This is a fair price to pay because any real-world program should call \mintinline{haskell}`fillLeaf` way more than \mintinline{haskell}`alloc` (which is only required to create the root of a new big object).

The implementation of \mintinline{haskell}`fromReg` and \mintinline{haskell}`fromRegExtract` is then relatively straightforward. \mintinline{haskell}`fromReg :: forall r a. (RegionContext r) => Incomplete r a () %1 -> Ur a` discards the \mintinline{haskell}`()` value carried by the Incomplete, allocates a \mintinline{haskell}`Ur _` in the region, writes the address of the \mintinline{haskell}`a` value into the \mintinline{haskell}`Ur` slot, and returns it. \mintinline{haskell}`fromRegExtract :: forall r a b. (RegionContext r) => Incomplete r a (Ur b) %1 -> Ur (a, b)` allocates a \mintinline{haskell}`Ur (_, _)` into the region, checks whether the \mintinline{haskell}`b` in \mintinline{haskell}`Ur b` is already part of the region or not (and copies it into the region if it isn't), and writes the address of the \mintinline{haskell}`a` and \mintinline{haskell}`b` values into the slots of the \mintinline{haskell}`Ur`ed pair before returning it.

\paragraph{Deriving \mintinline{haskell}`Fill` for all datatypes with `Generics`}

The \mintinline{haskell}`Fill liftedCtor r a` class has only one method, \mintinline{haskell}`fill :: Dest r a -> DestsOf liftedCtor r a`.

As it has been said before, the action of \mintinline{haskell}`fill @liftedCtor @r @a` is to allocate a new shallow constructor \mintinline{haskell}`Ctor _ :: a` in the region, and fill a destination of type \mintinline{haskell}`Dest r a` with its address. Because the constructor is shallow (i.e. its fields haven't been initialized), \mintinline{haskell}`fill` should also return destination objects pointing to these incomplete fields a.k.a. holes.

The type of the destinations that should be returned by \mintinline{haskell}`fill` can be computed by carefully inspecting the \mintinline{haskell}`Generic` representation of the type \mintinline{haskell}`a`. Indeed, GHC.Generics is a haskell library that provides compile-time inspection of a type metadata (list of constructors, fields, memory representation...) if the type implements the \mintinline{haskell}`Generic` class. Fortunately, the implementation of the \mintinline{haskell}`Generic` class be derived automatically by the compiler (instead of being implemented manually), making it easy to use for user-defined types.

In the \mintinline{haskell}`DestsOf` type family and in in the different \mintinline{haskell}`Fill` instance heads, I inspect the \mintinline{haskell}`Generic` representation of the given type \mintinline{haskell}`a` to find the metadata of the constructor whose lifted representation is \mintinline{haskell}`liftedCtor`, and then I extract the number of fields of the constructor and their types.

For example, the \mintinline{haskell}`Generic representation` of \mintinline{haskell}`Maybe a` gives

\begin{minted}[escapeinside=°°,linenos]{haskell}
M1 D (MetaData "Maybe" "GHC.Maybe" "base" False) (
  M1 C (MetaCons °\mnew{"Nothing"}° PrefixI False) °\mnew{U1}°
  :+: M1 C (MetaCons °\mnew{"Just"}° PrefixI False) (M1 S [...] °\mnew{(K1 R a)}°))
\end{minted}

indicating that there is zero fields with constructor `Nothing' and one field of type `a' with constructor `Just'. 


The representation for `(a, b)' gives

\begin{minted}[escapeinside=°°,linenos]{haskell}
  M1 D (MetaData "Tuple2" "GHC.Tuple.Prim" "ghc-prim" False) (
    (M1 C (MetaCons °\mnew{"(,)"}° PrefixI False) (
      M1 S [...] °\mnew{(K1 R a)}°
      :*: M1 S °\mnew{(K1 R b)}°))
\end{minted}

indicating that there are two fields, one of type `a' and one of type `b', with the `(,)' data constructor.

Fields of a constructor are stored contiguously in memory, at offset $i \times \textit{wordsize}_{(i \in 1..n)}$ from the constructor base address. So `fill' just harvests the field metadata and can easily build as many destinations as there are fields, using the address of the newly allocated constructor as a base point. The extracted types of the fields are used to specify the phantom type parameter of the destinations.

\subsection{Changes to GHC's internals and RTS}

The implementation we described in the previous parts relies on being able to allocate shallow constructors. This is the key point of destination-style programming: building structures in a top-down approach, where nodes deeper in the data tree are left unspecified for some time.

Usually, in Haskell, every field must be explicitly specified to allocate a constructor closure. This definitely makes sense as there is no (safe) way to later update a field in a mutable/in-place fashion; so an unspecified field would stay unspecified forever.

Of course, it is possible to overcome that well-founded rule by using a unique \emph{joker value} for every field of a constructor: `Right undefined' or `(undefined, undefined)' are valid haskell values that shouldn't blow up too fast. Using an unsafe coercion works too: `(unsafeCoerce (), unsafeCoerce ()) :: (a, b)' is valid. Theoretically, we hence have a way to allocate a somewhat shallow constructor.

That being said, there is no primitive in GHC at the moment that allows to allocate a value directly in a compact region. The only way to put something in a region is to copy something that has already been allocated in the normal GC heap. So even with the trick described above, the constructor closure would need to be allocated twice.

As a result, in order to obtain the maximum performance and memory gains from the use of destination, I had no other choice than to tweak the GHC compiler and add a new primitive operation (\emph{primop}) in charge of allocating a shallow constructor directly into a compact region.

\paragraph{About GHC architecture, RTS and primops}

GHC is the main compiler for the Haskell programming language, and its compilation model involves several intermediary languages.

First, Haskell code is turned into Core, a typed language whose syntax is mostly a subset of Haskell one. A lot of Haskell constructs are desugared into constructors and case expressions, type applications are made explicit, etc.
Then, several optimization steps are made on the Core output, like removing case expressions on known constructors, or floating `let` expressions (moving them upwards in the code to factor out some computations).

When all Core optimizations have been made, Core code is then transformed into STG code. In STG, all memory objects (called \emph{closures}) have the same memory representation, in the form of a pointer to a struct called the \emph{info table}, and a payload of a variable size (composed of primitive values and pointers to other closures). Because Haskell works in a lazy fashion, closures might not represent evaluated values, so the info table of each closure contains a pointer to a block of code that is responsible for evaluating the sub-expression that the closure represents. In other terms, the caller is not responsible for knowing how to evaluate the closure, but the closure itself is.
As you can imagine, having a dedicated info table with each closure would be excessively expensive memory-wise (as both of them would need to be dynamically allocated). Instead, the STG uses sharing as most as possible, and for example, every dynamic closure representing the constructor application of a given constructor (let's say `Just`) will have the same info table pointer, pointing to a single statically-allocated info table. Indeed, no matter what type of argument it is applied to, every `Just` instance will have the same memory layout (only one boxed argument as a payload), and the same evaluation code (doing nothing and returning immediately because a constructor application is already in weak-head normal form). That kind of sharing in fact quite common in other programming languages too: virtual tables are shared in the same way in C++.

Other specific info tables are also being shared in the STG: the ones for "blackholes" (which is used to indicate that a closure is currently under evaluation by a thread and that its result is still pending); the ones for indirection, etc.

All that sharing is made through the use of labels:

- when the declaration of a data constructor is encountered (definition site), the corresponding info table is emitted to the static area of the output program, and its address is associated with the label `<constructor name>_con_info`
- when the use of a data constructor is encountered, the label `<constructor name>_con_info` is used in place of an actual address in the closure that is emitted
- later on in the compilation process, those labels will be converted by the linker into actual addresses

Now, the runtime behaviour of a Haskell program is directed by the \emph{run-time system}, or RTS, which is a software component written in a mix of C and C-- (the last language in the compilation pipeline of Haskell for a native build). The RTS is built once for all when GHC itself is being built, and it is then included in every executable produced by GHC.

Its role is to manage threads, organize garbage collection and also to manage compact regions (among other things). The RTS defines various primops that allow the haskell programmer to interact with it for those needs. For example, `compactAdd#' is the primitive operation that copies a closure into an existing compact region, fully evaluating its potentiel sub-expressions along the way.

There are also primops in GHC which are meant to be resolved at the \emph{Core to Stg} phase (emitting some code at compile time), but which don't trigger any specific interaction with the RTS at runtime. Among those are all the primitive numeric operators on primitive types (`+#', `-#'...), and more generally all the performance-critical operations working on unboxed types, and which don't require any specific interaction or precaution with garbage collection or thread execution, or don't need a specific allocation scheme, etc.

It is important to note for the next of that part that the RTS cannot really access information that would have been made available during the compilation of a specific haskell program. Indeed, the RTS included in the final executable for the given program is the same as the one included in all other executables made with the same version of GHC ; it isn't possible to parametrize the RTS with data collected during the compilation of the given program (ex: types used or defined in the program). The only data channel I found between the compilation stage of a program and the RTS is to reify the needed compile-time information into a runtime value that will later be passed to the RTS through a primop call.

\paragraph{Find a good title here}

What I want to do is twofold:
- First, add a primitive to the RTS to allocate space into a compact region for a shallow constructor
- Second, add a non-RTS primitive to reify the necessary information about a constructor into a runtime-value so that it can be communicated to the RTS (because the RTS cannot access that itself, as I explained above)

- Story about getting the info table ptr or a given constructor (blogpost)

- How to allocate a shallow constructor

- Note about how to emulate that with gShallowCtor and "raw memory" runtime reflection when not using a custom GHC version

\section{Evaluating performance of destination-style programming}

\subsection{Benchmarking / profiling strategy}

- Three modes (-T, -s, -p)

- How to evaluate naive code (why we use both copyIntoReg and force)

\subsection{Performance of map implementation (in a strict context)}

- How this is important for Ocaml

\subsection{Performance of the SExpr parser}

\subsection{Performance of BFS Tree traversal}

\subsection{Qualitative evaluation of destination code VS naive code}

- TODO: implement a naive implement of functional mapMBFS

\section{Conclusion and related work}

- Why it's an improvement over Minamide

- Lifting the non-linear restriction for elements stored in dest-allocated structures (= requires more theoretical work)

- Using destinations in different contexts than compact regions (normal GC heap, other kinds of chunk-allocated memory)

\printbibliography

\end{document}
